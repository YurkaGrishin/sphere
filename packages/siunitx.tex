% \ProvidesFile{siunitx.cfg}
\sisetup{per-mode = symbol}
\sisetup{fraction-function = \sfrac}
\sisetup{inter-unit-product = \ensuremath{{}\cdot{}}}

% Основные единицы (SI base units, Table 1)

% Наименование                  Символ размерности Русское наименование Французское наименование Английское наименование Русское обозначение Международное обозначение
% Длина                         L                  метр                 mètre                    metre                   м                   m
% Масса                         M                  килограмм            kilogramme               kilogram                кг                  kg
% Время                         T                  секунда              seconde                  second                  с                   s
% Сила электрического тока      I                  ампер                ampère                   ampere                  А                   A
% Термодинамическая температура Θ                  кельвин              kelvin                   kelvin                  К                   K
% Количество вещества           N                  моль                 mole                     mole                    моль                mol
% Сила света                    J                  кандела              candela                  candela                 кд                  cd

\DeclareSIUnit\metre{\text{м}}
\DeclareSIUnit\meter{\text{м}}
%\DeclareSIUnit\kilogram{\text{кг}} % определяется через грамм
\DeclareSIUnit\second{\text{с}}
\DeclareSIUnit\ampere{\text{А}}
\DeclareSIUnit\kelvin{\text{К}}
\DeclareSIUnit\mole{\text{моль}}
\DeclareSIUnit\candela{\text{кд}}


% Производные единицы, имеющие специальные наименования и обозначения (Coherent derived units in the SI with special names and symbols, Table 2)

% Величина                                 Русское наименование Английское наименование Русское обозначение Международное обозначение Выражение через основные единицы
% Активность радиоактивного источника      беккерель            becquerel               Бк                  Bq                        с−1
% Температура Цельсия                      градус Цельсия       degree Celsius          °C                  °C                        K
% Электрический заряд                      кулон                coulomb                 Кл                  C                         А·с
% Электроёмкость                           фарад                farad                   Ф                   F                         Кл/В=с4·А2·кг−1·м−2
% Масса                                    грамм                gram                    г                   g                         10-3кг
% Поглощённая доза ионизирующего излучения грей                 gray                    Гр                  Gy                        Дж/кг=м²/c²
% Частота                                  герц                 hertz                   Гц                  Hz                        с−1
% Индуктивность                            генри                henry                   Гн                  H                         кг·м2·с−2·А−2
% Энергия                                  джоуль               joule                   Дж                  J                         Н·м=кг·м2·c−2
% Активность катализатора                  катал                katal                   кат                 kat                       моль/с
% Световой поток                           люмен                lumen                   лм                  lm                        кд·ср
% Освещённость                             люкс                 lux                     лк                  lx                        лм/м²=кд·ср/м²
% Сила                                     ньютон               newton                  Н                   N                         кг·м·c−2
% Сопротивление                            ом                   ohm                     Ом                  Ω                         В/А=кг·м2·с−3·А−2
% Давление                                 паскаль              pascal                  Па                  Pa                        Н/м2=кг·м−1·с−2
% Плоский угол                             радиан               radian                  рад                 rad                       м·м−1=1
% Электрическая проводимость               сименс               siemens                 См                  S                         Ом−1=с3·А2·кг−1·м−2
% Эффективная доза ионизирующего излучения зиверт               sievert                 Зв                  Sv                        Дж/кг=м²/c²
% Телесный угол                            стерадиан            steradian               ср                  sr                        м2·м−2=1
% Магнитная индукция                       тесла                tesla                   Тл                  T                         Вб/м2=кг·с−2·А−1
% Разность потенциалов                     вольт                volt                    В                   V                         Дж/Кл=кг·м2·с−3·А−1
% Мощность                                 ватт                 watt                    Вт                  W                         Дж/с=кг·м2·c−3
% Магнитный поток                          вебер                weber                   Вб                  Wb                        кг·м2·с−2·А−1

\DeclareSIUnit\becquerel{\text{Бк}}
%\DeclareSIUnit\degreeCelsius{\text{°C}}
\DeclareSIUnit\degreeCelsius{ \SIUnitSymbolCelsius }
\DeclareSIUnit\celsius{ \SIUnitSymbolCelsius }
\DeclareSIUnit\coulomb{\text{Кл}}
\DeclareSIUnit\farad{\text{Ф}}
\DeclareSIUnit\gram{\text{г}}
\DeclareSIUnit\gray{\text{Гр}}
\DeclareSIUnit\hertz{\text{Гц}}
\DeclareSIUnit\henry{\text{Гн}}
\DeclareSIUnit\joule{\text{Дж}}
\DeclareSIUnit\katal{\text{кат}}
\DeclareSIUnit\lumen{\text{лм}}
\DeclareSIUnit\lux{\text{лк}}
\DeclareSIUnit\newton{\text{Н}}
\DeclareSIUnit\ohm{\text{Ом}}
\DeclareSIUnit\pascal{\text{Па}}
\DeclareSIUnit\radian{\text{рад}}
\DeclareSIUnit\siemens{\text{См}}
\DeclareSIUnit\sievert{\text{Зв}}
\DeclareSIUnit\steradian{\text{ср}}
\DeclareSIUnit\tesla{\text{Тл}}
\DeclareSIUnit\volt{\text{В}}
\DeclareSIUnit\watt{\text{Вт}}
\DeclareSIUnit\weber{\text{Вб}}


% Единицы, не входящие в СИ (Non-SI units accepted for use with the International System of Units, Table 3)

% Единица         Английское наименование Русское обозначение Международное обозначение Величина в единицах СИ
% сутки           day                     сут                 d                         24ч=86400с
% угловой градус  degree                  °                   °                         (π/180)рад
% минута          minute                  мин                 min                       60с
% гектар          hectare                 га                  ha                        10000м²
% час             hour                    ч                   h                         60мин=3600с
% литр            litre                   л                   l,L                       0,001м³
% угловая минута  minute                  ′                   ′                         (1/60)°=(π/10800)
% угловая секунда second                  ″                   ″                         (1/60)′=(π/648000)
% тонна           tonne                   т                   t                         1000кг

\DeclareSIUnit\day{\text{сут}}
%\DeclareSIUnit\degree{\text{°}}
\DeclareSIUnit[ number-unit-product = ]\degree{ \SIUnitSymbolDegree }
\DeclareSIUnit\hectare{\text{га}}
\DeclareSIUnit\hour{\text{ч}}
\DeclareSIUnit\litre{\text{л}}
\DeclareSIUnit\liter{\text{л}}
\DeclareSIUnit [ number-unit-product = ] \arcmin { \arcminute }
%\DeclareSIUnit\arcminute{\text{′}}
\DeclareSIUnit [ number-unit-product = ] \arcminute { \SIUnitSymbolArcminute }
\DeclareSIUnit\minute{\text{мин}}
%\DeclareSIUnit\arcsecond{\text{″}}
\DeclareSIUnit [ number-unit-product = ] \arcsecond { \SIUnitSymbolArcsecond }
\DeclareSIUnit\tonne{\text{т}}


% Non-SI units whose values in SI units must be obtained experimentally, Table 4

\DeclareSIUnit\astronomicalunit{\text{а.е.}}
\DeclareSIUnit\atomicmassunit{\text{а.е.м.}}
\DeclareSIUnit\bohr{ \text { \ensuremath { a_{0} } } }
\DeclareSIUnit\clight{ \text { \ensuremath { c } } }
\DeclareSIUnit\dalton{\text{а.е.м.}}
\DeclareSIUnit\electronmass { \text { \ensuremath { m_{\textup{e}} } } }
\DeclareSIUnit\electronvolt{\text{эВ}}
%\elementarycharge
\DeclareSIUnit\hartree{ \text { \ensuremath { E_{\textup{h}} } } }
\DeclareSIUnit\planckbar{ \text { \ensuremath { \hbar } } }


% Other non-SI units, Table 5

% Единица      Английское наименование Русское обозначение Международное обозначение Величина в единицах СИ
% ангстрем     ångström                Å                   Å                         10−10м
% бар          bar                     бар                 bar                       100000 Па
% барн         barn                    б                   b                         10−28м²
% бел          bel                     Б                   B                         безразмерна
% узел         knot                    уз                  kn                        1 морская миля в час = (1852/3600) м/с
% морская миля nautical mile           миля                M                         1852 м (точно)
% непер        neper                   Нп                  Np                        безразмерна

\DeclareSIUnit\angstrom{\text{\AA}}
%\DeclareSIUnit\angstrom{\SIUnitSymbolAngstrom}
\DeclareSIUnit\are{\text{а}} % ар (100 м²) не имеет макроса в siunitx по умолчанию
\DeclareSIUnit\bar{\text{бар}}
\DeclareSIUnit\barn{\text{б}}
\DeclareSIUnit\bel{\text{Б}}
\DeclareSIUnit\decibel{\text{дБ}}
\DeclareSIUnit\knot{\text{уз}}
\DeclareSIUnit\mmHg{\text{мм\,рт.ст.}}
\DeclareSIUnit\nauticalmile{\text{миля}}
\DeclareSIUnit\neper{\text{Нп}}


% SI prefixes, Table 6

% Степень Русская приставка Международная приставка Русское обозначение Международное обозначение
% 1       дека              deca                    да                  da
% 2       гекто             hecto                   г                   h
% 3       кило              kilo                    к                   k
% 6       мега              mega                    М                   M
% 9       гига              giga                    Г                   G
% 12      тера              tera                    Т                   T
% 15      пета              peta                    П                   P
% 18      экса              exa                     Э                   E
% 21      зетта             zetta                   З                   Z
% 24      иотта             yotta                   И                   Y

\DeclareSIPrefix\deca{\text{да}}{1}
\DeclareSIPrefix\hecto{\text{г}}{2}
\DeclareSIPrefix\kilo{\text{к}}{3}
\DeclareSIPrefix\mega{\text{М}}{6}
\DeclareSIPrefix\giga{\text{Г}}{9}
\DeclareSIPrefix\tera{\text{Т}}{12}
\DeclareSIPrefix\peta{\text{П}}{15}
\DeclareSIPrefix\exa{\text{Э}}{18}
\DeclareSIPrefix\zetta{\text{З}}{21}
\DeclareSIPrefix\yotta{\text{И}}{24}


% Степень Русская приставка Международная приставка Русское обозначение Международное обозначение
% -1      деци              deci                    д                   d
% -2      санти             centi                   с                   c
% -3      милли             milli                   м                   m
% -6      микро             micro                   мк                  µ
% -9      нано              nano                    н                   n
% -12     пико              pico                    п                   p
% -15     фемто             femto                   ф                   f
% -18     атто              atto                    а                   a
% -21     зепто             zepto                   з                   z
% -24     иокто             yocto                   и                   y

\DeclareSIPrefix\deci{\text{д}}{-1}
\DeclareSIPrefix\centi{\text{с}}{-2}
\DeclareSIPrefix\milli{\text{м}}{-3}
\DeclareSIPrefix\micro{\text{мк}}{-6}
\DeclareSIPrefix\nano{\text{н}}{-9}
\DeclareSIPrefix\pico{\text{п}}{-12}
\DeclareSIPrefix\femto{\text{ф}}{-15}
\DeclareSIPrefix\atto{\text{а}}{-18}
\DeclareSIPrefix\zepto{\text{з}}{-21}
\DeclareSIPrefix\yocto{\text{и}}{-24}


% Степень Международное обозначение Международная приставка Русское обозначение Русское написание числа бит Русская приставка
% 10      kibi                      Ki                      киби                Кибит                       Ки
% 20      mebi                      Mi                      меби                Мибит                       Ми
% 30      gibi                      Gi                      гиби                Гибит                       Ги
% 40      tebi                      Ti                      теби                Тибит                       Ти
% 50      pebi                      Pi                      пеби                Пибит                       Пи
% 60      exbi                      Ei                      эксби               Эибит                       Эи
% 70      zebi                      Zi                      зеби                Зибит                       Зи
% 80      yobi                      Yi                      йоби                Йибит                       Йи

\DeclareBinaryPrefix\kibi{\text{Ки}}{10}
\DeclareBinaryPrefix\mebi{\text{Ми}}{20}
\DeclareBinaryPrefix\gibi{\text{Ги}}{30}
\DeclareBinaryPrefix\tebi{\text{Ти}}{40}
\DeclareBinaryPrefix\pebi{\text{Пи}}{50}
\DeclareBinaryPrefix\exbi{\text{Эи}}{60}
\DeclareBinaryPrefix\zebi{\text{Зи}}{70}
\DeclareBinaryPrefix\yobi{\text{Йи}}{80}



% Положение о единицах величин, допускаемых к применению в Российской Федерации,
% разрешает применение следующих внесистемных единиц:
% карат
% град (гон)
% световой год
% парсек
% фут
% дюйм
% килограмм-сила на квадратный сантиметр
% миллиметр водяного столба
% метр водяного столба
% техническая атмосфера
% диоптрия
% текс
% гал
% оборот в секунду
% оборот в минуту
% киловатт-час
% вольт-ампер
% вар
% ампер-час
% бит
% байт
% бит в секунду
% байт в секунду
% рентген
% бэр
% рад
% рентген в секунду
% кюри
% стокс
% калория (международная)
% калория термохимическая
% калория 15-градусная
% калория в секунду
% килокалория в час
% гигакалория в час

% Положение разрешает применять единицы относительных и логарифмических величин, такие как:
% процент
% промилле
% миллионная доля
% децибел
% фон
% октава
% декада

% Допускается также применять единицы времени, получившие широкое распространение, например:
% неделя
% месяц
% год
% век
% тысячелетие

% Не применяются с кратными и дольными приставками СИ наименования и обозначения внесистемных единиц:
% массы
% времени
% плоского угла
% длины
% площади
% давления
% оптической силы
% линейной плотности
% скорости
% ускорения
% частоты вращения

\DeclareSIUnit{\KWH}{\text{кВт\ensuremath{\cdot}ч}}
\DeclareSIUnit\percent {\text{\char 37}}
\DeclareSIUnit\curie    {\text{Ки}}
\DeclareSIUnit\gal      {\text{Гал}}
\DeclareSIUnit\rad      {\text{рад}}
\DeclareSIUnit\rem      {\text{Бэр}}
\DeclareSIUnit\roentgen {\text{Р}}
\DeclareSIUnit\parsec    {\text{пк}}
\DeclareSIUnit\lightyear {\text{св.год}}
\DeclareSIUnit\torr  {\text{торр}}
\DeclareSIUnit\gon    {\text{град}}
\DeclareSIUnit\gauss     {\text{Гc}}
\DeclareSIUnit\bit  {\text{бит}}
\DeclareSIUnit\byte {\text{Б}}